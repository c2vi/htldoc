%% Template basiert auf der Vorlage der Uni Graz für VWA: https://latex.tugraz.at/vorlagen/allgemein

%% Versionen:
%% V1: 9. Augugst 2021 (GreiA)


\input{template/settings}

%% header settings
\usepackage{lastpage}

\ohead{\headmark }
\ihead*{\includegraphics[width=3cm]{figures/htl-logo}}

\ifoot{\thepage}  %Will man Anzahl Seiten: /\pageref{LastPage}
\ofoot{\myauthor}


\input{template/mycommands}

\input{template/typographic_settings}
\input{template/listing_format}



\hyphenation{ex-am-ple hy-phen-ate}  %% in order to use German umlauts
%% here (Ver-\"of-fent-li-chung), you have to check for
%% activated \usepackage[T1]{fontenc} in the preamble


%% ========================================================================
%% bibtex für die Literaturverwaltung: Hier wird der Zitier-Stil festgelegt
%% ========================================================================
\usepackage{natbib} 
\bibliographystyle{agsm} 





\input{template/pdf_settings}  %% should be *last* definitions in preamble!

%% ========================================================================
%%%% begin{document}
%% ========================================================================

\begin{document} 
\frontmatter                    %% KOMA: roman page numbers and such; only available in scrbook

%% \input{colophon}                %% defines information about editor, LaTeX, font, ...

%% Choose your desired title page:
\input{\mytitlepage}            %% include title page

\input{template/lock_flag} % Wenn kein Sperrvermerk gemacht werden soll, dann diesen Import einfach auskommentieren

%%\input{template/declaration_TU_Graz}  %% Statutory Declaration
% \input{thanks}                %% this is a suggestion: you have to create this file on demand
% \input{foreword}              %% this is a suggestion: you have to create this file on demand


%% include the abstract without chapter number but include it on table of contents:
%%\let\cleardoublepage\clearpage
\phantomsection
\addcontentsline{toc}{chapter}{Abstract}
   
\chapter*{Kurzfassung /Abstract }
\label{cha:abstract}
\include{abstract}              %% Abstract generated from markdown

\input{template/affirmation} %%EIDESSTATTLICHE ERKLÄRUNG  

\tableofcontents                %% this produces the table of contents - you might have guessed :-)



%% if myaddlistoftodos is set to "true", the current list of open todos is added:
\ifthenelse{\boolean{myaddlistoftodos}}{
  \newpage\listoftodos          %% handy if you are using todonotes with \todo{}
}{}                             %% with todonotes-package option "disable" you can get rid of any todo in the output

\mainmatter                     %% KOMA: marks main part using arabic page numbers and such; only available in scrbook


%% HIER DIE EIGENEN KAPITEL EINFÜGEN
%\input{content/einleitung}  
%\input{content/latex_beispiele} % Einfach auskommentieren für die tatsächliche Arbeit

\appendix                       %% closes main document, appendix follows until end; only available in book-classes

\addpart*{Appendix}             %% adding Appendix to tableofcontents



\listoftables
\listoffigures
\lstlistoflistings
\nocite{*} %Es werden auf nicht referenzierte Literaturstellen aufgelistet
\bibliography{references}

\end{document}
